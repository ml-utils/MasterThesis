%   Based on the frontespizio (title page) template by Marco Antonio Corallo.

% Document type: the twoside setting makes chapters begin with their first page on the right. 
\documentclass[a4paper, twoside,openright]{report}

% Page margins options
\usepackage[a4paper,top=3cm,bottom=3cm,left=3cm,right=3cm]{geometry} 
% removes this restriction by allowing font sizes at arbitrary sizes
\usepackage{lmodern}
% font size
\usepackage[fontsize=13pt]{scrextend}
% Language(s) of the document contents
\usepackage[english]{babel} % ,italian
% Bibliography language
\usepackage[fixlanguage]{babelbib}

% Bibliography package
%Includes "References" in the table of contents
\usepackage[nottoc]{tocbibind}
% Bibliography package natbib, to have the "author-year" format for citations
\usepackage[round]{natbib} 
% \usepackage{biblatex}

% text encoding
\usepackage[utf8]{inputenc} 
\usepackage[T1]{fontenc}

% For rotating images
\usepackage{rotating}
% To change the header of pages
\usepackage{fancyhdr}               

% Math libraries
% \usepackage{amssymb}
\usepackage{amsmath}
\usepackage{amsthm}         

% For images
\usepackage{graphicx}
% For colors
\usepackage[dvipsnames]{xcolor}         
% Listing for code (?)
\usepackage{listings}          
% To insert hyperlinks 
\usepackage{hyperref}     
% For different types of underlinings
\usepackage[normalem]{ulem}

% \usepackage[superscript,biblabel]{cite}

% for tables, allows merging cells across multiple rows
\usepackage{multirow}
\usepackage{array}
\newcolumntype{L}{>{\centering\arraybackslash}m{1cm}}
\usepackage{booktabs}
\usepackage{siunitx}

% for multiple images in a figure
\usepackage{subcaption}

% for numbered lists:
\usepackage{enumitem}

% \usepackage{fullpage}
% \usepackage{amsmath,amssymb,amsthm}
% -----------------------------------------------------------------

% Changes the header style
\pagestyle{fancy}
\fancyhf{}
\lhead{\rightmark}
\rhead{\textbf{\thepage}}
\fancyfoot{}
\setlength{\headheight}{12.5pt}

% Removed the page number on the first page of a chapter
\fancypagestyle{plain}{
  \fancyfoot{}
  \fancyhead{}
  \renewcommand{\headrulewidth}{0pt}
}

% Syntax highlighting for programming code examples
\lstdefinestyle{codeStyle}{
    % Colore dei commenti
    commentstyle=\color{teal},
    % Colore delle keyword
    keywordstyle=\color{Magenta},
    % Stile dei numeri di riga
    numberstyle=\tiny\color{gray},
    % Colore delle stringhe
    stringstyle=\color{violet},
    % Dimensione e stile del testo
    basicstyle=\ttfamily\footnotesize,
    % newline solo ai whitespaces
    breakatwhitespace=false,     
    % newline si/no
    breaklines=true,                 
    % Posizione della caption, top/bottom 
    captionpos=b,                    
    % Mantiene gli spazi nel codice, utile per l'indentazione
    keepspaces=true,                 
    % Dove visualizzare i numeri di linea
    numbers=left,                    
    % Distanza tra i numeri di linea
    numbersep=5pt,                  
    % Mostra gli spazi bianchi o meno
    showspaces=false,                
    % Mostra gli spazi bianchi nelle stringhe
    showstringspaces=false,
    % Mostra i tab
    showtabs=false,
    % Dimensione dei tab
    tabsize=2
} \lstset{style=codeStyle}

% Stile di codice per dimensioni maggiori, in cui ho avuto bisogno di un testo più picolo (ad esempio se si vuole inserire del codice che ha linee molto lunghe). Per usare questo stile piuttosto che il precedente, usare 

% \lstset{style=longBlock}
%  % inserire il codice...
% \lstset{style=codeStyle}

% Il secondo comando consente di tornare allo stile precedente 
\lstdefinestyle{longBlock}{
    commentstyle=\color{teal},
    keywordstyle=\color{Magenta},
    numberstyle=\tiny\color{gray},
    stringstyle=\color{violet},
    basicstyle=\ttfamily\scriptsize,
    breakatwhitespace=false,         
    breaklines=true,                 
    captionpos=b,                    
    keepspaces=true,                 
    numbers=left,                    
    numbersep=5pt,                  
    showspaces=false,                
    showstringspaces=false,
    showtabs=false,                  
    tabsize=2
} \lstset{style=codeStyle}

% si inseriscono nella bibliografia anche le fonti presenti in Bibliography.bib ma non citati direttamente con il comando \cite
\nocite{*}

% Margini prima e dopo blocchi di codice, per avere più distanza
\lstset{aboveskip=20pt,belowskip=20pt}

% Modifica dello stile dei riferimenti, con il testo in cyano
\hypersetup{
    colorlinks,
    linkcolor=Blue,
    citecolor=Blue, % CornflowerBlue, % 
}

% Aggiunti definizioni, teoremi, linea e listing
\newtheorem{definition}{Definizione}[section]
\newtheorem{theorem}{Teorema}[section]
\providecommand*\definitionautorefname{Definizione}
\providecommand*\theoremautorefname{Teorema}
\providecommand*{\listingautorefname}{Listing}
\providecommand*\lstnumberautorefname{Linea}

\raggedbottom

%\newcommand{\cgs}[1]{{\textcolor{brown}[\textcolor{red}{\bf{GS: }}{ \textcolor{brown}{#1]}}}}             
%\newcommand{\cmc}[1]{{\textcolor{blue}[\textcolor{magenta}{\bf{MC: }}{ \textcolor{blue}{#1]}}}}


% -----------------------------------------------------------------
\begin{document}

\begin{titlepage}
\begin{figure}[!htb]
    \centering
    \includegraphics[keepaspectratio=true,scale=0.5]{images/Frontespizio/cherubinFrontespizio.eps}
\end{figure}

\begin{center}
    \LARGE{UNIVERSITÀ DI PISA}
    \vspace{5mm}
    \\ \large{Dipartimento di Filologia, Letteratura e Linguistica}
    \vspace{5mm}
    \\ \LARGE{Corso di Laurea Magistrale in Informatica Umanistica}
\end{center}

\vspace{15mm}
\begin{center}
    {\large{TESI DI LAUREA MAGISTRALE}}\\
\end{center}
\begin{center}
    {\LARGE{\bf Assessing island effects in Italian transformer-based language models }}
    
    % Se il titolo è abbastanza corto da stare su una riga, si può usare
    
    % {\LARGE{\bf Un fantastico titolo per la mia tesi!}}
\end{center}
\vspace{30mm}

\begin{minipage}[t]{0.47\textwidth}
	{\large{Relatore:}{\normalsize\vspace{3mm}
	\bf\\ \large{Prof: Alessandro Lenci}}}
\end{minipage}
\hfill
\begin{minipage}[t]{0.47\textwidth}\raggedleft
	{\large{Candidato:}{\normalsize\vspace{3mm} \bf\\ \large{Mauro Madeddu}}}
\end{minipage}

\vspace{30mm}
\hrulefill
\\\centering{\large{ANNO ACCADEMICO 2021/2022}}

\end{titlepage}
\cleardoublepage

% todo, fixme: the bibliography style turns titles all lowercase
\thispagestyle{plain}
\begin{center}
	\Large
	\vspace{1.8cm}
    \textbf{Abstract}
    	\vspace{0.9cm}
\end{center}

\normalsize
Modern language models based on deep artificial neural networks have achieved impressive progress in Natural Language Processing benchmarks and applications in the last few years. This has made increasingly important to clarify which linguistic phenomena and generalizations they actually learn. This has spawned a line of research on the fine-grained targeted linguistic evaluations of neural language models, in which the targeted syntactic evaluation approach in one of the main ones.

The assessment is done by administering to these models minimal pairs of sentences that vary minimally and isolate a particular linguistic phenomenon, and expect the model to give a higher score to the grammatical sentence over the ungrammatical one. A factorial experimental setup, common in psycholinguistic studies, can be considered a generalization of the minimal pairs approach, and allows to test more complex linguistic phenomena while still controlling for confounds.

% A generalization of the minimal pairs approach is the factorial experimental setup, imported from psycholinguistic research, in which test items are composed of more than two sentences, covering a combination of conditions, to better control for confounds, and allowing the assessment of more complex linguistic phenomena.

% or factorial items, minimally varying 
% administering to these models 
% testing these models' sentence acceptability estimates with fine-grained targeted linguistic evaluations, based on minimal pairs that isolate a particular linguistic phenomenon.\\

This kind of assessment is relevant to address the open problems of the limitations that these models still have, like being significantly data-inefficient in their training, compared to humans’ language acquisition and learning skills; or their still insufficient linguistic performance or generalization for some linguistic phenomena. This kind of assessment has also a broad interdisciplinary relevance since language models could be used to test theoretical linguistics hypotheses, and theoretical linguistics and psycholinguistics could in turn provide insights on how to improve these models' linguistic skills to more human-like levels.\\ 

In this work, we focus on the syntactic phenomena of island effects, and extend the Italian test suite from the psycholinguistic and experimental syntax work by \citet{sprouse2016experimental}. Then,  we evaluate on these test suites two transformer-based language models (Gpt-2 and Bert), pretrained in Italian, and compare their performance with those on humans. (..)% We propose that our adaptation of the factorial test design by Sprouse et al. could be complementary to benchmarks like BLiMP \citep{warstadt2020blimp} which are based on minimal pairs tests, and allows to test more aspects of linguistic phenomena like extraction islands.


\tableofcontents

\listoffigures
% \listoftables

\chapter{This is a chapter title}

..

\section{Accuracy results on island effects minimal pairs in Italian}

..

\subsection{Test description}

..

\subsection{Results table}


La tabella si indirizza sempre con l'uso di una label, ottenendo il risultato \autoref{tab:labelTabella}.

\begin{table}
	\caption{Una simpatica tabella!}\label{tab:labelTabella}
	\begin{center}
		\begin{tabular}{c|c|c|c}
			\textbf{Colonna 1} & \textbf{Colonna 2} & \textbf{Colonna 3} & \textbf{Colonna 4} \\
			\hline
			$3$      & $24$     & $24$    & $29$ \\ 
			$36$     & $31$     & $49$    & $39$ \\ 
			$32$     & $41$     & $59$    & $57$ \\ 
			$34$     & $60$     & $79$    & $74$ \\ 
			$328$    & $96$     & $194$   & $99$ \\ 
			$356$    & $117$    & $297$   & $149$ \\ 
			$312$    & $315$    & $293$   & $242$ \\ 
			$3024$   & $184$    & $253$   & $019$ \\ 
			$3048$   & $7795$   & $253$   & $077$ \\ 
			$3096$   & $7767$   & $2432$  & $0514$ \\ 
			$3192$   & $3769$   & $2435$  & $0551$ \\ 
			$36384$  & $6625$   & $3432$  & $0497$ \\ 
			$32768$  & $15469$  & $6472$  & $0471$ \\ 
			$35536$  & $15425$  & $14539$ & $10289$ \\ 
			$331072$ & $34623$  & $24941$ & $20444$ \\  
		\end{tabular}
	\end{center}
\end{table}

..


\subsection{Italian models details}

Bert (dbmdz/bert-base-italian-xxl-cased): 81GB (13 billion tokens) of training data  and from Wikipedia, OPUS and OSCAR corpora. Model size 424 MB.
GePpeTto (LorenzoDeMattei/GePpeTto): 13.8GB of training data from Wikipedia and ItWac corpus. The model’s size corresponds to GPT-2 small, with 12 layers and 117M parameters. Vocab size 30k. 620k training steps.
GilBERTo (idb-ita/gilberto-uncased-from-camembert): Trained on ~71GB of Italian text (11.2 billion tokens) from the OSCAR corpus. Model size 420 MB


\section{Factorial test design}

..

\subsection{Introduction}

..

\subsection{Results}


É possibile riferire l'immagine, una volta assegnatagli una label, tramite il comando \texttt{\textbackslash autoref\{fig:immagine\}}, ottenendo il seguente risultato: \autoref{fig:immagine}.

\begin{figure}
	\centering
	\includegraphics[width=0.8\textwidth]{images/Chapter1/immagine.jpg} % width= 0.8\textwidth
	\caption{This is an image} 
	\label{fig:immagine} % this internally labels the figure for future referencing.
\end{figure}

Fig.1 Results from Sprouse et al (2016)
Fig.2 Results of acceptability judgements to the same stimuli from Sprouse et al. 2016 from an Italian Gpt-2 model (GePpeTto) with the PenLP acceptability ..score. 
Fig.3

\subsection{Differences between the plots}

..
\subsubsection{Overall}
..
\subsubsection{Whether islands}
..
\subsubsection{Complex np islands}
..
\subsubsection{Subject islands}
..
\subsubsection{Adjunct islands}
..

\subsection{What seems to affect the models acceptability scores}
..
\subsubsection{Whether islands}
..
\subsubsection{Complex np islands}
..
\subsubsection{Subject islands}
..
\subsubsection{Adjunct islands}
..

\subsection{Future work}
..

\subsubsection{This is a sub-subsection}
..



Listing bibliography entries \citep{wei2021frequency, hu2020systematic, lau2020furiously,  sprouse2016experimental}

\subsubsection{This is another subsection}

Quello che segue è un esempio di codice. E' possibile modificare il linguaggio per il synyax highlight, aggiungere parole chiave... E' tutto disponibile nella guida del pacchetto \texttt{listings}.

\lstinputlisting[language=C++]{listings/Capitolo1/code1.cpp} 

\section{Blimp English dataset}
..
\subsection{English models details}
..

\section{Misc notes with refs}

“NATURAL LANGUAGE DOES NOT MAXIMIZE PROBABILITY” 
“Why is human-written text not the most probable text? We conjecture that this is an intrinsic property of human language. Language models that assign probabilities one word at a time without a global model of the text will have trouble capturing this effect. Grice’s Maxims of Communication (Grice, 1975) show that people optimize against stating the obvious. Thus, making every word as predictable as possible will be disfavored. This makes solving the problem simply by training larger models or improving neural architectures using standard per-word learning objectives unlikely: such models are forced to favor the lowest common denominator, rather than informative language.” 
\citep{holtzman2019curious}

Repeated exposure to a type of island construct will increase its perceived acceptability 
\citep{chaves2014subject}

Targed ..syntactic tests on modern language models seem to have started with \citet{linzen2016assessing}, while the use of psycholinguistic tests for this seem to have started with \citet{futrell2018rnns}.


\appendix

\chapter{Factorial design plots}

Models:
Bert: Scores obtained from the Huggingface Italian Bert model dbmdz/bert-base-italian-xxl-cased \footnote{https://huggingface.co/dbmdz/bert-base-italian-xxl-cased}

\clearpage
\section{Sprouse test suite}
\subsection{Bert}
\subsubsection{Measure: PenLP (from softmax model output)}
\begin{figure}[h]
	\centering
	\includegraphics[width=1\textwidth]{images/Chapter1/Sprouse_wh_dbmdz_bert-base-italian-xxl-cased_PenLP-zscores-likert-2022-07-11.png} 
\end{figure}

\clearpage
\subsubsection{Measure: LP (from softmax model output)}
\begin{figure}[h]
	\centering
	\includegraphics[width=1\textwidth]{images/Chapter1/Sprouse_wh_dbmdz_bert-base-italian-xxl-cased_LP-zscores-likert-2022-07-11.png} 
\end{figure}


\section{Madeddu test suite}
..

\bibliographystyle{agsm} 
\bibliography{chapters/Bibliography}


% .. 
% \addbibresource{references.bib}
% \include{chapters/Bibliografia}
\end{document}
% -----------------------------------------------------------------
