\thispagestyle{plain}
\begin{center}
	\Large
	\vspace{1.8cm}
    \textbf{Abstract}
    	\vspace{0.9cm}
\end{center}

\normalsize
Modern language models based on deep artificial neural networks have achieved significant progress in Natural Language Processing applications. This has spawned a line of research aimed at clarifying which linguistic phenomena and generalizations are actually learned by these models. One of the main approaches for this goal, is testing these models' sentence acceptability estimates with fine-grained targeted linguistic evaluations, based on minimal pairs that isolate a particular linguistic phenomenon.\\

This kind of assessment is relevant to address the open problems of the limitations that these models still have, like being significantly data-inefficient in their training, compared to humans’ language acquisition and learning skills; or their still insufficient linguistic performance or generalization for some linguistic phenomena. This kind of assessment has also a broad interdisciplinary relevance since language models could be used to test theoretical linguistics hypotheses, and theoretical linguistics and psycholinguistics could in turn provide insights on how to improve these models' linguistic skills to more human-like levels.\\ 

In this work, we focus on the syntactic phenomena of island effects, and extend the Italian test suite from the psycholinguistic and experimental syntax work by Sprouse et al. (2016). Then,  we evaluate on these test suites two transformer-based language models (Gpt-2 and Bert), pretrained in Italian, and compare their performance with those on humans. We propose that our adaptation of the factorial test design by Sprouse et al. could be complementary to benchmarks like BLiMP (Warstadt et al. 2020) which are based on minimal pairs tests, and allows to test more aspects of linguistic phenomena like extraction islands.

 %% Hello! this is the sourcecode!